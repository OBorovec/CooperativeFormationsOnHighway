\chapter{Conclusion}

Concept of platooning is known for long time, but is not still fully appreciated. We focused on overall positive effect of platooning and its potential for nowadays traffic in Czech highways from point of view of maximum capacity of highway and average speed. We did not deal with custom intern control of platoon itself, but we wanted to test cooperative approach to changing of lanes during various common action of travelling in highway such as overtaking.

For this purpose we upgraded an exist traffic simulator with many different adjustable features and we developed new module to be able to simulate platooning in highway. Result values from simulator “nearly” correspond to the theoretical model, and therefore we used it for simulation of traffic models, that represent required traffic situations. The simulated traffic experiments are based on real traffic statistic data that we gain by studying of several sources. The experiment parameters in combination with safety traffic setting gave rise of realistic traffic simulations which match a real traffic.

The experiments proved big theoretical potential of platooning for highway traffic. In same traffic composition (ration passenger vehicles and trucks), platooning concept can increase the maximum capacity of highway more than twice as it is nowadays. We also find out, that in cause application of platooning concept for actual traffic, it can increase average speed in 1-2 m/s. This positive effects were reached for 100\% platooning vehicles and were no proportional to percentage of platooning vehicles, but still there is a positive trend.

Cooperative behaviour of platoons had only small effect on maximum capacity of highway and average speed of nowadays traffic, but it dramatically decreases number of problematic situations, which endanger both platoon vehicles and the rest participants of traffic. Complexity of cooperative actions is not high but their benefit into safety of traffic is big. There is a smaller risk  of forced platoon decomposition of endanger of other vehicles.

Based on our research we could say that concept of platooning seems to be a positive step to improve our highway traffic environment. It has positive effect even on actual traffic level and also on increased number of vehicles and also on traffic safety. It can, according to global research, safe fuel and reduce amount of produced carbon dioxide.

In future steps it could be useful to test effect on traffic during day and run test based on all week data. There could be also analysed some other principles of cooperative actions such as other types of changing of lanes, clustering of platoon, etc. There is still a place to re-implement the PlatooningCenterModule to agent structure for better utilization of Alite simulator.

All requested assignment tasks were studied, done and evaluated.