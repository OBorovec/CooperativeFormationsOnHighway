\chapter{Introduction}
Everybody is grateful for Karl Benz's invention which saved us a lot of time thanks to the possibility to travel by cars instead of by carriages. But since the time of Henry Ford, who started serial cars production, the number of vehicles on our roads has been constantly increasing, which brings new problems and issues. The first way, how to increase the capacity of roads was the concept of highways but nowadays the density of traffic on the highways has reached such a level of use that sometimes it is better to drive along side roads. At this time a lot of drivers, cars, passengers, goods all over the world are stacked  in traffic jams although it probably is not necessary. 

We reckon that there are some ways how to deal with this situation. We can build new transport environment such as much bigger highways and a totally new aboveground concept of transport. Or we can increase car and transport speeds. The maximum speed of most new vehicles is higher than 200 km/h, but the increase of transport speed is not the direction how people can save their time, because the driving speed is one of the most frequent reasons of car accidents \footnote{\url{http://www.drivers.com/article/1173/}}. There is also another way and it is an optimization of current traffic system, which could be much cheaper and the modification would be up to the customers not up to the government's decision.

From the computer science point of view, we can see vehicles as agents which can cooperate with each other. These more autonomous vehicles can have more levels of driver support such as helping the drivers by useful advice or having a total control over the vehicles. Fully controlled vehicles have the advantage that, program is deterministic so, it does not make mistakes and does not succumb to external influences and panic. The most discussed problem is the law aspect, who is responsible for mistakes caused by the “autonomous” drivers. 

The cooperation of vehicles using Vehicle to Vehicle (V2V) and Vehicle to Infrastructure (V2I) protocols is already well-known and there are already some examples of products of BMW\footnote{\url{http://www.motorauthority.com/news/1064910_bmw-shows-off-its-v2v-technology-with-5-new-videos}}, General Motors\footnote{\url{http://media.gm.com/media/us/en/gm/news.detail.html/content/Pages/news/us/en/2014/Sep/0907-its-overview.html}}, and Toyota\footnote{\url{http://www.toyota-global.com/innovation/intelligent_transport_systems/world_congress/2014detroit/pdf/ITS_Integrated_Safety.pdf}} which we can see in the streets (for example Advanced driver assistance system which helps you in complicated situations in city or collision detection). The limitation of these improvements is that not every car (mostly the old ones and low cost) can cooperate with these systems.

Based on research and papers published by many international projects, we would like to make some partial research, to extract rules and conditions for vehicle cooperation and platooning and  experiment with them.

In this thesis we focus on research about simulating platoon transport concept that can optimize traffic and increase the capacity of transport network. We also focus on its applicability to some of the Czech highway traffic structures. The situation and some traffic statistics of the Czech traffic system are in Chapter 2, with emphasis on the composition of the traffic and on current pass ability.  In Chapter 3 we did an overview of existent papers and projects and we set basic rules of platoon concept. Based on these rules we developed and customised a Traffic simulator for platooning on highway, which is described in Chapter 4. With this simulator we prepared some test scripts and simulated some cases with the platooning concept settings. The tests were designed from the point of view of maximum capacity of highways and of platooning effect on actual traffic structure and can be found in Chapter 5. We tested the efficiency of the platooning concept with different ratio of cooperating and non-cooperating vehicles. And in the last part we simulated the concept on real data from the Czech highways to support the decision whether the concept is valuable for the Czech environment. We also created a protocol for platoon overtaking and tested its positive effect in Chapter 6.

